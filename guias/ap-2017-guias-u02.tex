\documentclass[11pt]{article}
\usepackage[spanish]{babel} 
\usepackage[utf8]{inputenc}
\usepackage{amsmath,amssymb}
\usepackage{color}
\renewcommand{\familydefault}{\sfdefault}

\textwidth 17cm
\textheight 25.8cm
\voffset -3.0cm
\hoffset -2.0cm

\begin{document}
\pagestyle{empty}
{Mayo 2017}
\begin{center}
	{\Large \bf Física de Astropartículas} \\
{ \large \bf Lluvias Atmosféricas Extendidas} \\ {\bf 1er semestre 2017}
\end{center}

\setcounter{enumi}{0}      %% Offset en numero de problema
\begin{enumerate}
	\item Un modelo más realista para el modelo de Heitler implica suponer que
		luego de cada capa atmosférica, cuyo espesor es
		$X_{EM}=37.1$\,g\,cm$^{-2}$, cada partícula produce $g_{\mathrm{EM}}$
		nuevas partículas. De esta forma, el modelo original de Heitler supone
		$g_{EM}=2$. Obtenga según este modelo realista los nuevos parámetros de
		la cascada, $N_{\max}$, $X_{\max}$ y $\Lambda$ ({\it{elongation
		rate}}), como función de $g_{EM}$. Luego, evalúe esos parámetros para
		los casos $g_{EM}=2, 6, 13, 20$.
	\item La atmósfera marciana es una mezcla de gases con la siguiente
		composición: $96\%$ de CO$_2$, $2\%$ de Ar, $1.8\%$ de N$_2$ y $0.2\%$
		de O$_2$. Es tan tenue que la presión atmosférica en su superficie es
		de sólo $6$\,hPa (como referencia, la presión atmosférica en la
		superficie terrestre es $1013.2$\,hPa), llegando a un máximo de
		$12$\,hPa en Hellas Planitia. Repita los cálculos del punto anterior
		para el caso de Hellas Planitia y compare los resultados obtenidos con
		los de la Tierra. Ayuda: algunos datos adicionales que podrían llegar a
		usar: masa de Marte: $6.42\times10^{23}$\,kg; radio de Marte:
		$3400$\,km. El número másico de una mezcla calcula simplemente como un
		promedio pesado por la fracción de cada constituyente, $\langle A
		\rangle = \sum_i A_i x_i$. Para el número atómico de la mezcla debe
		calcularse el número atómico efectivo,
		$Z_{\mathrm{eff}}=\sqrt[2.94]{\sum_i f_i Z_i^{2.94}}$, donde $Z_i$ es
		el número atómico del elemento $i$-ésimo y $f_i$ es la fracción de
		carga de cada elemento (es decir, $f_i=Z_i/\sum_i Z_i$). Por ejemplo,
		para el caso del agua, H$_2$O, $Z_{\mathrm{eff}}=7.42$.
	\item El modelo de Glasmaher-Matthews para cascadas hadrónicas es un modelo
		simple que permite comprender la evolución de una cascada iniciada por
		un hadrón. Hemos visto en clase el caso de una lluvia atmosférica
		extendida (EAS) iniciada por un protón de energía $E_p$, donde
		obtuvimos que la mayor parte de la energía se encuentra en el canal
		electromagnético ($E_{\mathrm{EM}} = 1 - (E_p/E_\pi)^{\beta_\pi - 1}
		\simeq 0.9$ para $\beta_\pi=0.85$). Luego, es válido suponer que a
		orden cero, $X_{\max} \simeq X_{\max}^{\mathrm{EM}}$, donde este último
		término corresponde a la posición del máximo de una cascada equivalente
		pero iniciada por un fotón de energía $E_\gamma=E_p / (3
		N_{\mathrm{CH}})$ (esto surge de suponer que en la primer interacción
		se producen $N_{\mathrm{CH}}$ y $N_{\mathrm{CH}}/2$, que decaen
		inmediatamente según la reacción $\pi^0 \to 2\gamma$, y luego
		$N_\gamma=N_{\mathrm{CH}}$). Bajo esta aproximación, demuestre que la
		posición del máximo para una cascada iniciada por un protón puede
		aproximarse cómo $X_{\max}^p = X_0 + X_{\max}^{\mathrm{EM}} -
		126$\,g\,cm$^{-2}$ para $N_{\mathrm{CH}}=10$ y $X_0$ corresponde al
		punto de primera interacción.
	\item Suponiendo que la energía límite $E_\pi$ es $E_\pi=10$\,GeV para
		piones cargados presentes en una cascada iniciada por un protón de
		$E_p=100$\,PeV, calcule a que altura sobre el nivel del mar debería
		incrementarse el número de muones en la cascada, suponiendo que el
		punto de primera interacción se dio en $X_0=0$\,g\,cm$^{-2}$. Considere
		y justifique todas las aproximaciones que considere necesarias para
		llegar al resultado pedido. Luego, a partir del poder de frenado del
		muón en el aire (vea Groom 2001), estime cuantos de esos muones
		llegaran a la superficie del suelo y cual con que energía promedio.
		Finalmente, use el poder de frenado del muón en roca estándar (Groom
		2001) para determinar cuantos de estos llegaran a un detector ubicado a
		$2.5$\,m bajo la superficie del suelo. En caso de requerirlo, modifique
		a voluntad y use el código {\texttt{stopping.py}} incluido en los
		materiales del curso. 
	\item El modelo de superposición para una EAS iniciada por un hadrón de
		masa $A$ y energía $E_p$ sostiene que la cascada resultante es
		equivalente a $A$ cascadas simultáneas iniciadas por sendos protones de
		energía $E_p/A$.  Este modelo se soporta en el hecho de que a las
		energías más altas, $E_p \gg m_p$, la energía de ligadura por nucleón
		(típicamente $B/A\simeq 8.8$\,MeV) es despreciable frente a $E_p/A$.
		Utilice entonces el modelo de Glasmaher-Matthews combinado con el
		modelo de superposición para verificar que:
		\begin{enumerate}
			\item El número de muones de una lluvia iniciada por un núcleo de
				hierro ($^{56}$Fe$_{26}$) es $\simeq 50\%$ mayor que el número de
				muones de una lluvia iniciada por un protón (estrictamente,
				$N_\mu^A \simeq A^{1-\beta_\pi} N_\mu^p$).
			\item La posición del máximo puede aproximarse como $X_{\max}^A =
				X_{\max}^{p} - X_{\mathrm{EM}} \ln A$.
			\item Las fluctuaciones en la posición del máximo de distintas
				lluvias con la misma energía del primario son menores para los
				hierros que para los protones. 
		\end{enumerate}
	\item Calcule el número de muones que esperaría ver para una lluvia
		iniciada por un protón con energía $E_p = 100$\,PeV. Repita sus
		cálculos para un helio, un carbono, un oxigeno, un calcio y un hierro
		de la misma energía.
	\item A partir de la definición del radio de Molière, 
		\[ X_m = \sqrt{\frac{4\pi}{\alpha_{\mathrm{EM}}}} \frac{m_e
		c^2}{E_{c}^{\mathrm{EM}}} X_{\mathrm{EM}}\]
		calcule el radio de Moliere, $r_m$ en metros en los siguientes casos:
		a) La Tierra a nivel del Mar; b) La Tierra en Chacaltaya ($h=5350$\,m
		s.n.m); c) La Tierra en altura de vuelo ($h=11500$\,m s.n.m.); d) Marte
		en la superficie.
	\item En una lluvia iniciada por un núcleo de masa $A$, el número de
		electrones puede calcularse como 
		\[N_e^A \simeq 7.35\ A^{-0.046} \left (
		\frac{E_p}{10^{17}\mathrm{\,eV}} \right )^{1.046} \times 10^7.\]
		Luego, el número de muones queda parametrizado como:
		\[N_\mu^A \simeq 0.95 \times 10^5  \left ( \frac{N_e^A}{10^6} \right
		)^{0.75}.\]
		A partir de esto, Greisen propuso una parametrización para la LDF de
		los muones, que aún se utiliza, con algunos cambios. El experimento
		KASCADE utilizó esta parametrización para la densidad de muones a una
		distancia $r$ al eje de la lluvia, medida sobre el frente de la misma:
		\[
			\rho_\mu(\varrho_\mu) = \left (\frac{k_G N_\mu}{r_G^2} \right )
			\varrho_\mu^{-0.69} \left ( 1 + \varrho_\mu \right
			)^{-2.39} \left ( 1 + \left (0.1 \varrho_\mu \right )^2 \right
			)^{-1.0}
		\]
		donde $N_\mu$ es el número total de electrones, $k_G=0.28$ es una
		constante de normalización y $\varrho_\mu\equiv r/r_G$ es la distancia
		normalizada al radio de Molière para muones, también llamado radio de
		Greisen, $r_G=320$\,m. Estas expresiones completan las vistas en
		clase para los electrones, 
		\[
			\rho_e(\varrho, S) = \left ( \frac{\Gamma(4.5 - S)}{2 \Gamma(S)
			\Gamma(4.5 - 2S)} \right ) \left( \varrho \right)^{S-2} \left(1 +
			\varrho \right)^{S-4.5} \left ( \frac{N_e(S)}{\pi R_M^2} \right),
		\]
		donde $\Gamma$ es la extensión del factorial a números reales y
		complejos,
		\[
			\Gamma(z) = \int_0^\infty e^{-t} t^{z-1} \mathrm{d} t \,\, \to \,\,
			\Gamma(n) = (n-1)! \,\,\mathrm{si}\,\, n \in \mathbb{Z}^+,
		\]
		$\varrho \equiv r/R_m$ es la distancia $r$ al eje expresada en unidades
		del radio de Molière, y $N_e(S)$ es el número total de electrones a la
		edad $S$, dado por
		\[
			N_e (S) = 0.31 y^{-1/2}  \exp \left [t \left ( 1 - \frac32 \ln S
			\right ) \right ],
		\]
		que continúa siendo válida para un hadrón si se corrige la edad de la
		lluvia por $S_A=1.15\ S$.
		
		Utilice todas estas expresiones según corresponda para calcular el
		número de electrones y muones y su distribución a nivel del suelo, y
		compárelas con los resultados obtenidos en las simulaciones de un
		fotón, un protón y un hierro verticales con energía $E_p=0.5$\,PeV
		realizadas en clase para un observatorio situado en la ciudad de la
		furia.
		
		Luego, si se anima, calcule la profundidad del máximo $X_{\max}$ y la
		altura del máximo $h_{\max}$ para un fotón vertical de $2$\,PeV.
		Entonces posicione un nivel de observación en CORSIKA a dicha altitud y
		realice la simulación de un fotón, un protón y un hierro de procedencia
		vertical con esa energía. Repita los cálculos y comparaciones para el
		número y la distribución de electrones y muones en cada caso. 
\end{enumerate}


\end{document}
