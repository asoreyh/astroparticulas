\documentclass[11pt]{article}
\usepackage[spanish]{babel} 
\usepackage[utf8]{inputenc}
\usepackage{amsmath}
\usepackage{amssymb}
\usepackage{color}
\renewcommand{\familydefault}{\sfdefault}

\textwidth 17cm
\textheight 25.8cm
\voffset -3.0cm
\hoffset -2.0cm

\begin{document}
\pagestyle{empty}
{Junio 2017}
\begin{center}
	{\Large \bf Física de Astropartículas} \\
{ \large \bf Detectores - B} \\ {\bf 1er semestre 2017}
\end{center}

\setcounter{enumi}{0}      %% Offset en numero de problema
\begin{enumerate}
	\item {\bf{Telescopios Cherenkov}}\\
		Con el fin de estudiar la producción Cherenkov durante el desarrollo de
		una EAS, partiendo de la distribución LAGO-CORSIKA compile una versión
		de CORSIKA, desactivando la versión curva (-7a) y activando la opción
		CERENKOV (opción 1a y luego subopciones 1 y 1) y CEFFIC (opción 1c).
		Utilice como ejemlo para los {\textit{inputs}} de CORSIKA el archivo
		{\texttt{cherenkov.input}}, editándolo de acuerdo a lo solicitado. Entonces:
		\begin{description}
			\item [Previo a la simulación]: considere la lectura del manual de
				CORSIKA para los datacards {\texttt{CERARY, CERQEF, CERSIZ,
				CERFIL, CWAVLG}}. En particular, {\texttt{CERARY}} crea un arreglo
				rectangular de detectores Cherenkov en el nivel de observación,
				donde se contarán los fotones Cherenkov producidos durante la
				EAS. Por ejemplo, {\texttt{CERARY 201 201 10.E2 10.E2 2.E2
				2.E2}} creará una cuadrícula de 201x201 detectores, centrados
				en $(0,0)$, con una superficie de (2x2)\,m$^2$, y separados por
				$10$\,m en la dirección $X$ y $10$\,m en la dirección $Y$.
			\item [Simulación:] simule la producción Cherenkov para un fotón,
				un protón y un hierro, con dirección vertical y energías
				$E_p/\mathrm{GeV} = 2\times10^1, 2\times10^2, 2\times10^3,
				2\times10^4, 2\times 10^5$, para un arreglo como el mencionado
				anteriormente situada en San Antonio de Los Cobres ($h=3700$\,m
				s.n.m., $BX=20.94$\,$\mu$T y $BZ=-8.91$\,$\mu$T, y atmósfera
				MODTRAN subtropical estándar en verano (E2). Es importante
				recordar que deben cambiar las semillas de las simulaciones en
				antes de cada corrida. Al finalizar la simulación, verá que
				además del consabido archivo DATnnnnnn, tendrá también un
				archivo CERnnnnnn conteniendo la distribución de fotones
				detectados por el arreglo rectangular.
			\item [Análisis:] al igual que el archivo {\texttt{DATnnnnnn}}, el
				archivo {\texttt{CERnnnnnn}} es un archivo fortran binario sin
				formato. El mismo puede ser leído y transformado en ASCII con,
				por ejemplo, el código {\texttt{lagocrkread}}:\\
				{\texttt{echo CERnnnnnn | ./lagocrkread > CERnnnnnn.dat}}\\
				Luego, el archivo ASCII resultante puede ser analizado con el
				código {\texttt{cherenkov}}, provisto junto con el paquete
				ARTI. El mismo espera como parámetro la línea completa que
				comienza con {\texttt{CERARY}} del archivo {\texttt{input}}:\\ 
				{\texttt{./cherenkov -v -c CERARY 201 201 10.E2 10.E2 2.E2 2.E2 CERnnnnnn.dat}}\\
				Con esto, produce un histograma bidimensional contenido en el archivo
				{\texttt{CERnnnnnn.hst}} con el número de fotones y el número
				de {\textit{hits}} que impactan sobre los detectores (recordar
				que los fotones Cherenkov son simulados en CORSIKA en forma de
				paquetes conteniendo un número variable de fotones. Los
				{\textit{hits}} cuentan el número de paquetes que impactan).
			\item [Resultados:] Analice los histogramas obtenidos en cada caso
				y explique todas y cada una de las diferencias observadas en
				función de lo visto en clase.  
		\end{description} 
\end{enumerate}
\end{document}
