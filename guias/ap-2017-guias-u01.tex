\documentclass[11pt]{article}
\usepackage[spanish]{babel} 
\usepackage[utf8]{inputenc}
\usepackage{amsmath,amssymb}
\usepackage{color}
\usepackage[colorlinks=true,urlcolor=blue,linkcolor=blue, citecolor=blue]{hyperref}
\renewcommand{\familydefault}{\sfdefault}

\textwidth 17cm
\textheight 25.0cm
\voffset -3.0cm
\hoffset -2.0cm

\begin{document}
\pagestyle{empty}
{Mayo 2017}
\begin{center}
	{\Large \bf Física de Astropartículas} \\
{ \large \bf Astropartículas} \\ {\bf 1er semestre 2017}
\end{center}

\setcounter{enumi}{0}      %% Offset en numero de problema
\begin{enumerate}
	\item Suponiendo que el espectro de rayos cósmicos (RC) sigue una ley de
		potencias de la forma $j(E)=j_0 E^{\alpha}$, calcule el número total de
		rayos cósmicos que arriban a la Tierra por año y por km$^2$ en los
		siguientes rangos de energía:
		\begin{enumerate}
			\item $10^{7} \leq E/\mathrm{GeV} \leq 10^{8}$ con $\alpha=-3.3$.
			\item $10^{7} \leq E/\mathrm{GeV} \leq 10^{8}$ con $\alpha=-3.0$.
			\item $10^{3} \leq E/\mathrm{GeV} \leq 10^{4}$ con $\alpha=-2.7$.
		\end{enumerate}
		en todos los casos obtenga el valor de $j_0$ de los espectros
		publicados (ver p. ej. espectro en U01).
	\item Verifique que la fuerza de Lorentz relativista puede escribirse en
		forma covariante como $$ \frac{dp^\mu}{d\tau} = q F^{\mu\nu} u_\nu$$
		donde $p^\mu$ es el 4-momento, $p^\mu = (\gamma m c, p_x, p_y, p_z)$,
		$\tau$ es el tiempo propio de la partícula, $F^{\mu\nu}$ es la forma
		contravariante del tensor de Maxwell y $u_\nu$ es la forma covariante
		de la 4-velocidad, $u_\nu= \gamma (c, -v_x, -v_y, -v_z)$. Notar que se
		usó la métrica usual en partículas, $\eta=\mathrm{diag}(1,-1,-1,-1)$.
	\item Demuestre que el radio de Larmor de una partícula de masa $m$ y
		carga $q$ que se mueve en presencia de un campo magnético $\vec B$ con
		velocidad $\vec v$ formando un ángulo $\theta$ con el campo magnético
		puede escribirse como $$ r= \frac{\gamma m v \sin \theta}{|q|
		B}.$$Luego, haciendo los cambios de unidades que considere necesarios,
		pruebe que la expresión anterior puede reescribirse como $$r = 3.3 
			\left ( \frac{\gamma m c^2}{\mathrm{GeV}} \right )
			\left ( \frac{v_\perp}{c} \right )
			\left ( \frac{e}{|q|} \right )
			\mathrm{\ \ metros}.
		$$ Finalmente, imagine una partícula de carga $q=Ze$, masa $m$ y
		cantidad de movimiento $p$, moviéndose en las inmediaciones del campo
		magnético de:
		\begin{enumerate}
			\item la Tierra;
			\item un magnetar;
			\item la heliósfera;
			\item la Vía Láctea;
			\item el lóbulo de un AGN.
		\end{enumerate}
		En cada caso, calcule la energía $E_p$ y la rigidez magnética, $R = p c
		/ q$, de forma tal que el radio de Larmor de la partícula sea del orden
		del tamaño de cada uno de esos objetos. Se considera, a orden cero, que
		esa es la capacidad máxima de aceleración o de confinamiento de cada
		una de esos objetos. Utilice la bibliografía recomendada o el material
		de clase para obtener los datos faltantes, y evalúe las expresiones
		halladas para el caso de un protón, de un núcleo de carbono y de un
		núcleo de hierro.
	\item Usando las variables de Mandelstam, y en particular
		$s=E_{\mathrm{CM}}^2$, verifique que la energía de la colisión en el
		LHC ($13$\,TeV) es igual a $\sim 10^{5}$\,TeV en el sistema de
		laboratorio (una de las partículas está en reposo, aire).
	\item El fondo de radiación cósmica (CMB, por sus siglas en inglés) se
		ajusta de una manera extraordinaria con el espectro de un cuerpo negro
		a una temperatura de $T=(2.718\pm0.027)$\,K (ver p. ej,
		\href{https://arxiv.org/abs/1502.01589}{arXiv:1502.01588[astro-ph.CO]}).
		A partir de los resultados del Planck en 2015, calcule el valor medio y
		el valor más probable de la energía de los fotones del CMB. Luego,
		calcule la densidad numérica de fotones del CMB (ayuda: recuerde que
		para un cuerpo negro la densidad de energía es $(4\sigma/c)T^4$, donde
		$\sigma$ es la constante de Stefan-Boltzmann y $c$ es la velocidad de
		la luz).
	\item Suponiendo que la capacidad de una fuente le permite acelerar
		protones hasta una energía de corte $E_c = 4\times10^{15}$\,eV.
		Calcule el espectro combinado (H,He,C,Fe) de la fuente suponiendo que
		el flujo de 1-Hidrógeno es $\mathcal{F}_{\mathrm{H}} = (1.15\times10^{-5})
		E^{-2.77}\,\mathrm{m}^{-2}\,\mathrm{sr}^{-1}\,\mathrm{s}^{-1}\,\mathrm{TeV}^{-1}$,
		el flujo de 4-Helio es  $\mathcal{F}_{\mathrm{He}} = (7.19\times10^{-6})
		E^{-2.64}\,\mathrm{m}^{-2}\,\mathrm{sr}^{-1}\,\mathrm{s}^{-1}\,\mathrm{TeV}^{-1}$,
		el flujo de 12-Carbono es $\mathcal{F}_{\mathrm{C}} = (1.06\times10^{-6})
		E^{-2.66}\,\mathrm{m}^{-2}\,\mathrm{sr}^{-1}\,\mathrm{s}^{-1}\,\mathrm{TeV}^{-1}$, 
		y el flujo de 56-Hierro es $\mathcal{F}_{\mathrm{Fe}} = (1.78\times10^{-6})
		E^{-2.6}\,\mathrm{m}^{-2}\,\mathrm{sr}^{-1}\,\mathrm{s}^{-1}\,\mathrm{TeV}^{-1}$.
	\item Siguiendo los lineamientos de Protheroe\&Clay, 2004, verifique que
		en el mecanismo de Fermi de 2do orden predice un incremento medio de
		energía $\langle \Delta E \rangle \simeq 4/3 \beta^2 E$ y un espectro
		del tipo ley de potencias $J(E) \propto E^{\alpha}$ con $\alpha < -1$.
		Luego describa los principales inconvenientes de este modelo. Repita
		lo anterior para el caso del mecanismo de Fermi de primer orden
		($\langle \Delta E \rangle \simeq 4/3 \beta E$ y $\alpha \simeq -2$).
	\item Usando el invariante de Mandelstam $s$, verifique los umbrales de
		energía de los siguientes procesos:
		\begin{description}
			\item[Fotoproducción de piones] $p^+ + \gamma_{\mathrm{CMB}} \to
				p^+ + \pi^0$, $E_{p^+} \gtrsim 30$\,EeV
			\item[Fotonucleoproducción de piones] $A + \gamma_{\mathrm{CMB}}
				\to A + \pi^0$, $E_{A} \gtrsim 30 \left ( 1 + m_\pi / \left ( A
				m_p \right ) \right )$\,EeV. 
			\item[Fotoproducción de pares] $p^+ + \gamma_{\mathrm{CMB}} \to p^+
				+ e^+ + e^-$, $E_{p^+} \gtrsim 3$\,EeV
			\item[Fotonucleoproducción de piones] $A + \gamma_{\mathrm{CMB}}
				\to A + e^+ + e^-$, $E_{A} \gtrsim 3 \left ( 1 + m_e / \left (
				A m_p \right ) \right )$\,EeV.
		\end{description}
	\item Calcule la longitud de decaimiento ($\lambda_\tau=\beta\gamma\tau c$)
		para un $\pi^0$ con energía $E=1$\,PeV.
	\end{enumerate}
\end{document}
