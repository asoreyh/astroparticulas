\documentclass[11pt]{article}
\usepackage[spanish]{babel} 
\usepackage[utf8]{inputenc}
\usepackage{amsmath}
\usepackage{amssymb}
\usepackage{color}
\renewcommand{\familydefault}{\sfdefault}

\textwidth 17cm
\textheight 25.8cm
\voffset -3.0cm
\hoffset -2.0cm

\begin{document}
\pagestyle{empty}
{Junio 2017}
\begin{center}
	{\Large \bf Física de Astropartículas} \\
{ \large \bf Detectores - A} \\ {\bf 1er semestre 2017}
\end{center}

\setcounter{enumi}{0}      %% Offset en numero de problema
\begin{enumerate}
	\item Imagine que se desea construir un detector Cherenkov, pero en lugar
		de usar agua, se usará disulfuro de carbono (CS$_2$), que tiene un
		índice de refracción mayor que el agua y que podemos suponer constante,
		i.e., $n_{\mathrm{CS}_2}(\lambda) \equiv n = 1.627$, y con una densidad
		algo mayor también, $\rho_{\mathrm{CS}_2}(\lambda) \equiv \rho =
		1.3$\,g\,cm$^{-3}$. Para construir al detector se utilizará una esfera
		de radio $r=1$\,m, y en algún punto de su superficie se montará un PMT
		con la siguiente eficiencia cuántica:
		$$
			\mathrm{QE} \equiv \frac{\mathrm{fotoelectrones\
			producidos}}{\mathrm{fotones\ incidentes}} = 
			\begin{cases}
				0    & \lambda < 250\mathrm{\ nm} \\
				0.30 & 250\mathrm{\ nm} \leq \lambda \leq 600\mathrm{\ nm} \\
				0    & \lambda > 600\mathrm{\ nm}.
			\end{cases}
		$$
		Utilizando como guía las expresiones calculadas en clase, y la curva
		del poder de frenado para electrones en CS$_2$ (ver tabla), calcule:
		\begin{enumerate}
			\item El ángulo máximo de emisión Cherenkov $\theta_{\mathrm{Ch}}$
				en este líquido. 
			\item El umbral de producción Cherenkov $\beta_{\mathrm{Ch}}$, y el
				correspondiente momentum $p$, energía cinética $K$ y energía
				total $E$ que deben tener electrones, muones y protones para
				ser detectados.  Luego, calcule la energía mínima que debe
				tener un fotón para ser detectado mediante el proceso de
				creación de pares en el CS$_2$.
			\item La curva de producción de fotones Cherenkov como función de
				la energía de la partícula (vea como ejemplo la figura de la
				diapositiva 26/66 de la clase U03-detectores-a), para
				electrones, muones y protones.
			\item El valor de $X_EM$ y la energía crítica $E_c$
				electromagnética. Compare este valor con la energía más
				probable de los electrones que ingresan al detector para las
				lluvias simuladas en la unidad 2. Luego, calcule el factor de
				atenuación de la esfera en la dirección radial para esos
				electrones.
			\item Estime si para este detector esférico, es necesario incluir
				algún material difusor en la superficie interna del mismo con
				el fin de uniformizar la distribución de fotones Cherenkov.
			\item A partir del rango estimado para electrones (ver tabla),
				calcule, cuando corresponda, el número total de fotones
				Cherenkov producidos por la propagación de un electrón con
				energía $E=\{0.5; 5; 50; 500\}$\,MeV (por simplicidad, suponga
				que la curva de producción de fotones es una función escalón,
				que vale 0 por debajo de la energía umbral de producción y el
				valor de saturación por encima. Haga y describa las
				aproximaciones que considere necesarias para estimar el total
				de fotones.)
		\end{enumerate}
		\begin{center}
			\begin{tabular}{|c|c|c|c|c|}
				\hline
				\multicolumn{5}{c}{ESTAR: Stopping Powers and Range Tables for
				Electrons} \\
				\multicolumn{5}{c}{Carbon disulfide
				$\rho=1.2927$\,g\,cm$^{-3}$, Ionization=175.9\,eV} \\
				\hline
				$K$ (MeV) & $S_{\mathrm{Col}}$ (MeV cm$^2$/g) &
				$S_{\mathrm{Rad}}$ (MeV cm$^2$/g) & $S_{\mathrm{Tot}}$ (MeV
				cm$^2$/g) & Range (g/cm$^2$)\\
				\hline
				5.000E-01 & 1.653E+00 & 1.425E-02 & 1.668E+00 & 2.186E-01 \\ 
				5.000E+00 & 1.635E+00 & 1.440E-01 & 1.779E+00 & 2.961E+00 \\ 
				5.000E+01 & 1.900E+00 & 1.995E+00 & 3.895E+00 & 1.942E+01 \\ 
				5.000E+02 & 2.093E+00 & 2.255E+01 & 2.465E+01 & 5.958E+01 \\
				\hline
			\end{tabular}
		\end{center}
	\item Para un WCD cilíndrico de radio $r$ y altura $h$ obtenga una función
		$T(\theta_\mu)$ que de la distancia recorrida por un muón que ingrese
		al detector con un ángulo $\theta_\mu$ respecto a la vertical de tal
		manera que su trayectoria intercepta al eje de simetría del cilindro.
		\begin{enumerate}
			\item Definamos como $1$\,VEM a la señal producida en el WCD por el
				pasaje de un muón central y vertical, y que dicha señal es
				proporcional a la distancia recorrida por la partícula en el
				interior del detector. Utilizando la función $T(\theta_\mu)$,
				calcule la distribución de señal esperada como función de
				$\theta_\mu$ para los siguientes factores geométricos del WCD:
				$d/h=1$, $d/h=2$ y $d/h=3$. Este último caso corresponde al
				detector del arreglo SD de Auger. Luego, calcule para este
				detector ($h=1.2$\,m y $d=3.6$\,m) la distribución de señal
				esperada.
			\item A partir de la curva de señal esperada, calcule la LDF$_\mu$
				de la señal esperada de muones para detectores tipo SD a partir
				de los datos simulados de las cascadas de la unidad 2.
			\item Para estimar la señal de la componente EM, puede utilizar la
				siguiente relación válida para electrones:
				$$\log_{10} E_d(p) = \sum_{i=0}^2 a_i \log_{10}
				\left(\frac{p}{\mathrm{MeV/c}} \right)^i,$$
				donde $a_2=-0.131, a_1=1.337$ y $a_0=-0.29$, $p$ es la cantidad
				de movimiento del electrón y $E_d$ es la energía depositada en
				el detector (y la calibración indica que $1$\,VEM=$200 h$\,MeV,
				siendo $h$ la altura en m). Para los fotones puede verse que
				en promedio depositan la misma energía que los electrones pero
				con una probabilidad de conversión del $\sim 80$\% para $E_\gamma
				\geq 2$\,MeV, y $\sim 0\%$ para energías menores. Grafique la
				función anterior y luego compárela con la aproximación calorimétrica
				($E_d \simeq E$), i.e., grafique la fracción $\xi(E) =
				\frac{E-E_d}{E}$ como función de la energía del secundario $E$
				en el rango entre $1$\,MeV$\lesssim E \lesssim 500$\,MeV
			\item Utilice la expresión anterior para calcular la
				LDF$_{\mathrm{EM}}$ a partir de la distribución de la
				componente EM de las simulaciones realizadas.
			\item Dado que en el orden de las aproximaciones realizadas la
				componente hadrónica puede ser despreciada, calcule la LDF
				total como la suma de las dos contribuciones anteriormente
				mencionadas.
		\end{enumerate}
\end{enumerate}
\end{document}
